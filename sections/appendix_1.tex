\chapter{Tables}
\label{appendix:tables}

\begin{sidewaystable}[h]
\centering
\begin{tabularx}{\textwidth}{X X X X}
\hline
\rowcolor[HTML]{C0C0C0} 
\multicolumn{1}{|X|}{\cellcolor[HTML]{C0C0C0}\textbf{Viewpoint}} & \multicolumn{1}{X|}{\cellcolor[HTML]{C0C0C0}\textbf{Stakeholder}} & \multicolumn{1}{X|}{\cellcolor[HTML]{C0C0C0}\textbf{Description}}                                                                                                              & \multicolumn{1}{X|}{\cellcolor[HTML]{C0C0C0}\textbf{Applicable Model Types (UML)}}                             \\ \hline
\multicolumn{1}{|X|}{Logical view}                               & \multicolumn{1}{X|}{End-Users}                                    & \multicolumn{1}{X|}{Breakdown of the objects or parts of a system. It describes the functionality to be provided to the end-users}                                             & \multicolumn{1}{X|}{Class Diagrams, State Diagrams, Object Diagrams, Sequence Diagrams,Communication Diagrams} \\ \hline
\multicolumn{1}{|X|}{Process view}                               & \multicolumn{1}{X|}{Administrators and Managers}                  & \multicolumn{1}{X|}{Provides information on the workflow of objects into processes in order to explain the run-time behaviour of the system and how its processes communicate} & \multicolumn{1}{X|}{Activity Diagram}                                                                          \\ \hline
\multicolumn{1}{|X|}{Physical view}                              & \multicolumn{1}{X|}{System Engineer, Architect, Developer}        & \multicolumn{1}{X|}{Overview of the physical hardware and software the system is composed of, how they are arranged (topology) and how they communicate.}                      & \multicolumn{1}{X|}{Deployment Diagram}                                                                        \\ \hline
\multicolumn{1}{|X|}{Development view/ Implementation view}      & \multicolumn{1}{X|}{Developer, software Engineers}                & \multicolumn{1}{X|}{Provides overview of project structure from programming viewpoint including libraries, packages and run-time environments}                                 & \multicolumn{1}{X|}{Component Diagram, Package diagrams}                                                       \\ \hline
\multicolumn{1}{|X|}{Use case viewpoint/ Scenario viewpoint}     & \multicolumn{1}{X|}{All stakeholders}                             & \multicolumn{1}{X|}{Functional requirements of the system broken down into scenarios that indicate uses of the system by its stakeholders}                                     & \multicolumn{1}{X|}{Use Case Diagrams}                                                                         \\ \hline
                                                                 &                                                                   &                                                                                                                                                                                &                                                                                                               
\end{tabularx}
\caption{4+1 Viewpoint to UML Mapping}
\label{table:viewpoints}
\end{sidewaystable}



\section{ISO Definitions}
\begin{table}[h]
\centering
\begin{tabularx}{\textwidth}{|p{2.5cm}|X|l}
\cline{1-2}
\cellcolor[HTML]{C0C0C0}\textbf{Term} & \cellcolor[HTML]{C0C0C0}\textbf{Definition}                                                                                                                                              &  \\ \cline{1-2}
Architecting                          & Process of conceiving, defining, expressing, documenting, communicating, certifying proper implementation of, maintaining and improving an architecture throughout a system's life-cycle &  \\ \cline{1-2}
Architecture                          & Fundamental concepts or properties of a system in its environment embodied in its elements, relationships, and in the principles of its design and evolution                             &  \\ \cline{1-2}
Architecture Description (AD)         & The work product used to express an architecture                                                                                                                                         &  \\ \cline{1-2}
Architecture Framework                & Conventions, principles and practices for the description of architectures established within a specific domain of application and/or community of stakeholders                          &  \\ \cline{1-2}
Architecture View                     & Work product expressing the architecture of a system from the perspective of specific system concerns                                                                                    &  \\ \cline{1-2}
Architecture Viewpoint                & Work product establishing the conventions for the construction, interpretation and use of architectural views to frame specific system concerns                                          &  \\ \cline{1-2}
Architectural Concern                 & Interest in a system relevant to one or more of its stakeholders                                                                                                                         &  \\ \cline{1-2}
Environment                           & context determining the setting and circumstances of all influences upon a system                                                                                                        &  \\ \cline{1-2}
\end{tabularx}
\caption{ISO 42010 Definitions}
\label{table:iso_def}
\end{table}
\newpage


\section{Quality Attribute Scenarios}
\begin{table}[h]
\centering
\begin{tabularx}{\linewidth}{|
>{\columncolor[HTML]{EFEFEF}}l |X|l}
\cline{1-2}
\multicolumn{2}{|l|}{\cellcolor[HTML]{C0C0C0} Scalability: System loads requires scaling horizontally} &  \\ \cline{1-2}
Element & \cellcolor[HTML]{EFEFEF}Scenario &  \\ \cline{1-2}
Stimulus & 

Requires specific nodes/components to accommodate more users 
& \\ \cline{1-2}
Source & 

System owner 
& \\ \cline{1-2}
Environment & 

Normal Operation, Heavy System Loads 
&  \\ \cline{1-2}
Artefact & 

Specific system node/component 
& \\ \cline{1-2}
Response & 

Scale specific node/component horizontally for duration of the heavy load and scale back once load is reduced below certain threshold 
&  \\ \cline{1-2}
Measure & 

Scale by metric 
&  \\ \cline{1-2}
\end{tabularx}
\caption{QAR1 Scenario}
\label{table:qar1}
\end{table}

\begin{table}[h]
\centering
\begin{tabularx}{\linewidth}{|
>{\columncolor[HTML]{EFEFEF}}l |X|l}
\cline{1-2}
\multicolumn{2}{|l|}{\cellcolor[HTML]{C0C0C0} Availability: Application Node Failure} &  \\ \cline{1-2}
Element & \cellcolor[HTML]{EFEFEF}Scenario &  \\ \cline{1-2}
Stimulus & 

System node/component has a fault 
& \\ \cline{1-2}
Source & 

Fault (External or Internal), Datacenter failure
& \\ \cline{1-2}
Environment & 

Normal, Datacenter Maintenance, Degraded System Mode
&  \\ \cline{1-2}
Artefact & 

System 
& \\ \cline{1-2}
Response & 

In event of one of the system component failures all operations should be directed to alternative running instances. System should spin up new instances if required 
&  \\ \cline{1-2}
Measure & 

Downtime interval 
&  \\ \cline{1-2}
\end{tabularx}
\caption{QAR2 Scenario}
\label{table:qar2}
\end{table}

\begin{table}[h]
\centering
\begin{tabularx}{\linewidth}{|
>{\columncolor[HTML]{EFEFEF}}l |X|l}
\cline{1-2}
\multicolumn{2}{|l|}{\cellcolor[HTML]{C0C0C0} Modifiability: Adding Tenants} &  \\ \cline{1-2}
Element & \cellcolor[HTML]{EFEFEF}Scenario &  \\ \cline{1-2}
Stimulus & 

Wants to add a tenant to the system without custom views
& \\ \cline{1-2}
Source & 

Tenant Manager, Administrator, Developer
& \\ \cline{1-2}
Environment & 

Normal operations on Front-end
&  \\ \cline{1-2}
Artefact & 

System 
& \\ \cline{1-2}
Response & 

The system adds the new tenant to the tenant data and allows the tenants domain to access the system using the default views 
&  \\ \cline{1-2}
Measure & 

Effort to add tenant 
&  \\ \cline{1-2}
\end{tabularx}
\caption{QAR3 Scenario}
\label{table:qar3}
\end{table}


\begin{table}[h]
\centering
\begin{tabularx}{\linewidth}{|
>{\columncolor[HTML]{EFEFEF}}l |X|l}
\cline{1-2}
\multicolumn{2}{|l|}{\cellcolor[HTML]{C0C0C0} Tenants performance isolation} &  \\ \cline{1-2}
Element & \cellcolor[HTML]{EFEFEF}Scenario &  \\ \cline{1-2}
Stimulus & 

Utilizes a large number of resources or load creates a load spike
& \\ \cline{1-2}
Source & 

Tenant
& \\ \cline{1-2}
Environment & 

System under high loads
&  \\ \cline{1-2}
Artefact & 

System and Services
& \\ \cline{1-2}
Response & 

Other tenants should not be noticeably affected by a single tenants increased loads. Little to no impact on front-end responsiveness 
&  \\ \cline{1-2}
Measure & 

Front-end latency
&  \\ \cline{1-2}
\end{tabularx}
\caption{QAR4 Scenario}
\label{table:qar4}
\end{table}


\begin{table}[h]
\centering
\begin{tabularx}{\linewidth}{|
>{\columncolor[HTML]{EFEFEF}}l |X|l}
\cline{1-2}
\multicolumn{2}{|l|}{\cellcolor[HTML]{C0C0C0} Modifiability: Adding Tenants with Custom Views} &  \\ \cline{1-2}
Element & \cellcolor[HTML]{EFEFEF}Scenario &  \\ \cline{1-2}
Stimulus & 

Wants to add a tenant to the system with custom views
& \\ \cline{1-2}
Source & 

Tenant Manager, Administrator, Developer
& \\ \cline{1-2}
Environment & 

Normal operations on Front-end
&  \\ \cline{1-2}
Artefact & 

System
& \\ \cline{1-2}
Response & 

The developers create custom views that correspond to the tenants requirements and do a redeployment, after QAR3 is executed the system should then use the custom views over the default views for the tenant with the specified tenant configuration
&  \\ \cline{1-2}
Measure & 

Effort of provisioning new tenant with custom views
&  \\ \cline{1-2}
\end{tabularx}
\caption{QAR5 Scenario}
\label{table:qar5}
\end{table}

\begin{table}[h]
\centering
\begin{tabularx}{\linewidth}{|
>{\columncolor[HTML]{EFEFEF}}l |X|l}
\cline{1-2}
\multicolumn{2}{|l|}{\cellcolor[HTML]{C0C0C0} Security: Tenant Data Isolation} &  \\ \cline{1-2}
Element & \cellcolor[HTML]{EFEFEF}Scenario &  \\ \cline{1-2}
Stimulus & 

Wants to ready tenant specific data from the respective data store
& \\ \cline{1-2}
Source & 

Tenant Advertisers, Tenant Managers
& \\ \cline{1-2}
Environment & 

Normal operation
&  \\ \cline{1-2}
Artefact & 

System and System Data
& \\ \cline{1-2}
Response & 

The system should isolate other tenant data and ensure that no unauthorized access to other tenants data is made and no modifications of other tenant data takes place
&  \\ \cline{1-2}
Measure & 

Tenant Data Leakage (0\% allowed)
&  \\ \cline{1-2}
\end{tabularx}
\caption{QAR6 Scenario}
\label{table:qar6}
\end{table}


\begin{table}[h]
\centering
\begin{tabularx}{\linewidth}{|
>{\columncolor[HTML]{EFEFEF}}l |X|l}
\cline{1-2}
\multicolumn{2}{|l|}{\cellcolor[HTML]{C0C0C0} Usability: Modify Front-end per Tenant} &  \\ \cline{1-2}
Element & \cellcolor[HTML]{EFEFEF}Scenario &  \\ \cline{1-2}
Stimulus & 

Wants to adapt the system UI to specific tenant workflows and requirements
& \\ \cline{1-2}
Source & 

Tenant Manager
& \\ \cline{1-2}
Environment & 

Run-time, Configure Time under normal operation
&  \\ \cline{1-2}
Artefact & 

System, Front-End
& \\ \cline{1-2}
Response & 

System UI workflow and requirements should be settable through configuration switches that are dynamically loaded and manipulate the systems UI, workflow and usability
&  \\ \cline{1-2}
Measure & 

Ease of use, level of system adaptability
&  \\ \cline{1-2}
\end{tabularx}
\caption{QAR7 Scenario}
\label{table:qar7}
\end{table}



\section{Architectural Drivers \& Priorities}
% Please add the following required packages to your document preamble:
% \usepackage[table,xcdraw]{xcolor}
% If you use beamer only pass "xcolor=table" option, i.e. \documentclass[xcolor=table]{beamer}
\begin{table}[h]
\centering
\begin{tabularx}{\linewidth}{lllX}
\hline
\rowcolor[HTML]{EFEFEF} 
\multicolumn{1}{|l|}{\cellcolor[HTML]{EFEFEF}Driver} & \multicolumn{1}{l|}{\cellcolor[HTML]{EFEFEF}Importance} & \multicolumn{1}{l|}{\cellcolor[HTML]{EFEFEF}Impact} & \multicolumn{1}{l|}{\cellcolor[HTML]{EFEFEF}Elements} \\ \hline
\rowcolor[HTML]{FFCCC9} 
FR2 & High & High & Presentation \\
\rowcolor[HTML]{FFCCC9} 
QAR2 & High & High & Infrastructure, Presentation, Application, Persistence \\
\rowcolor[HTML]{FFCCC9} 
QAR3 & High & High & Presentation \\
\rowcolor[HTML]{FFCCC9} 
DC2 & High & High & Infrastructure, Presentation, Application, Persistence \\
\rowcolor[HTML]{FFCCC9} 
QAR1 & High & High & Infrastructure, Presentation, Application, Persistence \\
\rowcolor[HTML]{FFCE93} 
QAR6 & Medium & High & Presentation, Infrastructure \\
\rowcolor[HTML]{FFCE93} 
DC1 & Medium & High & Infrastructure \\
\rowcolor[HTML]{FFCE93} 
FR10 & Medium & High & Persistence \\
\rowcolor[HTML]{FFCE93} 
FR5 & Medium & High & Presentation, Persistence \\
\rowcolor[HTML]{FFFC9E} 
QAR5 & High & Medium & Persistence \\
\rowcolor[HTML]{FFFC9E} 
QAR4 & High & Medium & Application \\
\rowcolor[HTML]{F9F9E4} 
FR11 & Medium & Medium & Presentation \\
\rowcolor[HTML]{F9F9E4} 
DC3 & Low & Medium & Presentation, Application, Infrastructure \\
\rowcolor[HTML]{CAF6C9} 
FR1 & High & Low & Presentation, Persistence \\
\rowcolor[HTML]{CAF6C9} 
FR3 & High & Low & Presentation, Application \\
\rowcolor[HTML]{CAF6C9} 
FR6 & High & Low & Presentation \\
\rowcolor[HTML]{CAF6C9} 
FR7 & High & Low & Presentation \\
\rowcolor[HTML]{CAF6C9} 
FR4 & Medium & Low & Presentation \\
\rowcolor[HTML]{CAF6C9} 
FR8 & Medium & Low & Presentation \\
\rowcolor[HTML]{CAF6C9} 
DC5 & Medium & Low & Presentation \\
\rowcolor[HTML]{CAF6C9} 
FR9 & Low & Low & Presentation \\
\rowcolor[HTML]{CAF6C9} 
DC4 & Low & Low & Presentation, Application
\end{tabularx}
\caption{Architectural Drivers, Priorities and Impact}
\label{table:architecturaldrivers}
\end{table}
\begin{center}
\textbf{Note}: FR = Functional Requirement, DC = Design Constraint, \\ QAR = Quality Attribute Requirement. For more information on specifics of these, please see section \ref{sec:reqandconstraints} on page~\pageref{sec:reqandconstraints}
\end{center}
\newpage


\section{Scalability patterns}
\begin{table}[h]
\centering
\begin{tabularx}{\textwidth}{|p{2cm}|X|X|l}
\cline{1-3}
\cellcolor[HTML]{EFEFEF}Pattern & \cellcolor[HTML]{EFEFEF}Advantages & \cellcolor[HTML]{EFEFEF}Disadvantages &  \\ \cline{1-3}
CQRS & Clear separation of Command and Query responsibilities that allows either to be scaled independently. Asynchronous commands with delayable execution and direct storage querying & Added Complexity. Commands can not return updated values. Forces eventual consistency and deeper domain knowledge &  \\ \cline{1-3}
Queue Centric Workflow & Immense loose coupling.  Removes concurrency issues.Asynchronous execution of time consuming commands. Clean separation of concerns. & Delayed execution time. Could have bottleneck effect if not properly adjusted for synchronous  scalability with workload &  \\ \cline{1-3}
Command Pattern & Allows using commands passed as messages in QCW. Allows mechanism for implementing asynchronous execution of commands in CQRS & Creation of many specific command classes &  \\ \cline{1-3}
\end{tabularx}
\caption{Scalability Patterns}
\label{tab:scalability_patterns}
\end{table}

