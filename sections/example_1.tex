\chapter{Introduction} 

The advent of cloud computing has changed the way we look at Service Science. With the rapidly declining cost of Infrastructure, Platform and Software as a Service offerings and the increased potential of cloud technologies many are moving away from a classic self hosting infrastructure to an cloud alternative. One of the biggest advantages that cloud computing provides potentially limitless vertical scalability. As a result, Software Engineers and Computer Architects have needed to change the way they design and implement their software and systems. This meant stepping back from a single instance, single tenant model and considering utilising the advantages the cloud has provided by means of multi-tenancy.

This study focuses on first understanding what multi-tenancy is and why it is relevant to the design of cloud based solutions. A case study is used to provide a relevant context to the research from which design considerations are drawn and finally an theoretical architecture implementing multi-tenancy for the specific case requirements is derived. 

\section{Problem}
Classical software or service design has been by and large single tenant, single instance. Applications are designed and modified for a specific client or tenant. The application is then hosted for that tenant and any or all changes required by the tenant is implemented for that application instance. If the same application needs to be modified for another tenant the application is modified and another instance of the modified application is hosted. This allows for the same application core to be hosted in many instances, each serving a specific tenant. Although this model allows for some particular benefits, it does not fit in the current 'massive scalability' and 'service based' cloud paradigm. Provisioning of new tenants is a manual process and scalability is mostly vertical (higher performance of computer hosting the instance) compared to cloud specialised horizontal scaling (more instances sharing application workload). Furthermore, vertical scaling is usually unidirectional, providing no elasticity. Modifications or implementation of new features also need to be implemented for each individual tenants code base in cases where a separate code base is used. In cases where a single code base is used, switches and checks are overused in order to determine the application behaviour. This dramatically adds to application complexity and reduces maintainability. 

\subsection{Background}
The gap between classical hosted and cloud paradigms has given the need for developing multi-tenant solutions. These multi-tenant solutions implement a single application instance in order to serve multiple application tenants while allowing each tenant to extend, configure or modify the application as if they were the sole tenant. Such a multi-tenant system usually come at a cost of higher design and implementation complexity. However this increased complexity is usually trivial in comparison to the long term benefits such as the application scalability, availability and maintainability. Add to this the financial benefits gained by economies of scales, pay per usage models and the reduced cost of commodity hardware, multi-tenant solutions are a powerful alternative for companies looking to gain differential benefit. This is the reason the study introduces exactly such a company as case study. The utilization of a specific case study with its surrounding contexts allow us to obtain a set of design inputs that will be used for defining the architectural constraints and requirements. The most important of these being the limitation for our investigation to using Windows Azure as our cloud platform as well as other Microsoft related products and services.  Although research and discussions on multi-tenancy are still limited, three major works of literature are used as the foundation for this study. Firstly is Microsofts guide on building Multi-tenant application for the cloud using Windows Azure \cite{Betts2012-ad}. The guide provides a strong theoretical discussions and considerations to take during application design. Although this literature helps one to understand the exact considerations to take when designing the architecture, many of the technologies suggested and used within the guide are now obsolete or have been replaced by better alternatives. This is one of the major influencers for this work. Specific design patterns for the cloud are also fundamental to the construction of such an multi-tenant architecture and are thoroughly discussed \cite{Wilder2012-so}. Finally, methods for addressing the exact specific challenges as set out by our case study is measured through application of Prescriptive Architecture Guidance for Cloud Applications \cite{Homer2014}. 

\subsection{Statement of the thesis}
\begin{fancyquotes}
This paper aims to create an software architecture as artifact that is scalable, customizable and multi-tenant efficient. The artifact should also address the issues and concerns outlined by the case study and be designed around its specific requirements
\end{fancyquotes}

\subsection{Statement of the problem}
In relation to the discussed problem, this paper aims to answer three questions relating to multi-tenancy with varying degrees of specificity:
\begin{enumerate}
\item Which are the currently available Azure technologies that can be used to create, architect and implement an multi-tenant solution?
\item What generic application architecture can be used to implement an multi-tenant applications for a variety of cloud platforms?
\item How can an Azure Specific multi-tenant architecture be implemented as prototype artifact to solve the major requirements for our case study?
\end{enumerate}

\section{About this study}
\subsection{Methodology}
This study relies on design science as its research as it is primarily a problem solving paradigm \cite{Hevner2004a}. It is characterized by the creation of a research artifact in the form of models, methods or instantiations that enable researchers to understand the problems related to the development and implementation of IT and information systems \cite{March1995a}. These artifacts are created by the extension of existing kernel theories that are then applied, tested and modified through the experience, creativity and intuition of the researcher \cite{Walls1992}. In many cases design science is applied in order to solve identified organiational problems. In such cases, the resulting artifact allows researchers to understand the efficiency and feasibility of the artifact as solution to such a problem \cite{Hevner2004a}. 

In this study we use qualitative means to assess and evaluate means of design, implementation and improvement of the artifact.  This evaluation provides feedback and allows one to understand the problem better in order to improve the artifact and its design process continuously. It is common for the design artifact to continuously evolve as research is conducted and new theories or directions are undertaken \cite{Hevner2004a}. 

The artifact which is the primary aim of the research is an design model. Hevner et. al (2004) defined models as a set of connections between problem and solution components that enable exploration of the effects of design decisions and changes in the real world. Therefore the creation of an 'multi-tenant architectural model' can be seen as the principal goal of the research.  The secondary supporting artifact which is created as means to show the feasibility, allow assessment of and prove validity of the primary artifact is an instantiation artifact \cite{Hevner2004a} in the form of a prototype. The prototype aims to implement the minimal features outlined by the primary artifact in order as merely supporting element. The completeness of effectiveness of the design artifact is evaluated in relation to its completeness in addressing the requirements of the original problem as well as its validity to the use case context \cite{Hevner2004a}. It is important to keep in mind that since design science is an iterative approach starting with a simplified conceptual presentation of the problem which is assessed in a changing technological and organisational environment that research assumptions are subject to change \cite{Johansson2000}.

\subsection{Relevance}
As most research revolving cloud technologies are still relatively new and continuously changing, it is hard to establish true relevance of a research topic for an extended period. In order to contribute to the pool of knowledge research should provide results that could be perpetually used in order to improve the deep knowledge in current trends at an appropriate speed to their development. Therefore this research claims relevance in the current interest in cloud technologies and their ability to allow multi-tenant systems to be used as 'cloud ready' alternative architecture. Thus the artifact produced as research results should address the observational requirements of the case study providing a relevant model considering current technologies which could be used either by organisations wishing to implement a multi-tenant system with a similar context to the case study or other researchers that wish to extend the model for other technologies, use cases or problem domains. Finally, scientific research should be evaluated in light of its actual practical implementations, proving its relevance to the field and as research  \cite{Hevner2004a}. This paper aims to evaluate its practicality through observation and application to the research case study.

\subsection{Aim}
This study focuses on Multi-tenant application design using \\ Microsoft Azure as cloud platform in order to determine a solid application architecture that would resolve the major requirements of the crossvertise case study. All this is done in order to better understand the role, design and implementation of multi-tenant applications for the cloud.

\subsection{Objectives}
The research problem is approached by firstly evaluating current and related cloud literature. Context to the research problem is then provided through means of the Crossvertise XV-platform use case. This use case is used as primary design input from which requirements, constraints and considerations are taken whereafter an in depth dissection of available technologies provide by Microsoft Azure is done. A broad overview of relevant design patterns is given and taken into account before creation of the initial architectural artifact or model. The model is then used to extract a general model for implementation of an multi-tenant application and the Azure specific model is then used to implement an simplistic prototype which is used as 'proof of concept' and validity for the architectural model.

\subsection{Assumptions}
Assumptions are generally circumstances which are taken as an absolute while being outside of the control of the researcher \cite{simon2010dissertation}. In this study some of these assumptions include:
\begin{itemize}
\item Microsoft Azure is a publicly available cloud platform
\item The high level architectural design concepts produces are generalisable to other cloud providers and technologies with refinement and alteration
\end{itemize}

\subsection{Limitations}
Limitations are potential problems in a research project that are outside the researchers control  \cite{simon2010dissertation}. Some if the identified limitations of this project include:
\begin{itemize}
\item The results of this study is very technology dependent and therefore cannot be completely generalised for SaaS providers using any platform. It only serves as foundation for understanding how a multi-tenant system can be implemented in Azure and should be used for gaining context. In order to address this limitation, an high level general architecture has been extracted from the results
\item The prototype produced in this study is aimed only as acting as proof of concept for the architectural design discussed in this research paper and does not aim to be a complete implementation of our case study platform. It should mimic some fundamental functionality of the case study platform only to show suggested means of implementing the different concepts of the architecture. Furthermore, the prototype has a few critical limitations discussed in the final chapter concerning hosting due to the high costs of using the services that the actual case study platform would require
\end{itemize}

\subsection{Delimitations} \footnote{The lists for limitations, delimitation and assumptions should be extended}
Delimitations are a set of characteristics that help to limit the scope and clearly define the boundaries of a study \cite{simon2010dissertation} and as such have been defined as:
\begin{itemize}
\item This research paper does not aim to implement a fully fledged commerce platform for the purpose of selling advertising media. Instead, it uses this domain as context for designing an architecture that could be used to build such a system
\item Technological delimitation are made according to the \\ case study e.g. the usage of Microsoft Azure, ASP.NET MVC and C\# for building the platform
\end{itemize}


\section{Research Outline} \footnote{Clearly define the research outline once all the chapters have been finalised}
Chapter 1 introduces the problem domain of this paper and aims to familiarise the reader with the background to the problem and its relevance for investigation. It also clearly explains the intent of the research and outlines any limitations, assumptions and delimitation that are made. The next chapter introduces us to to multi-tenancy and its implementation in the cloud. It also examines different levels of SaaS maturity in order to provide grounds for the necessity for SaaS providers to consider multi-tenancy as a crucial part of their software design architecture.
