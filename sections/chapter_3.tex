\chapter{Methodology}

In this chapter an overview of the methodologies, paradigm and process used in order to construct our application architecture is given. It briefly introduces design science as research paradigm and explains how this paper uses design science in order to produce a model as artefact. Furthermore, the use of case study as tool is discussed and justified. Finally, the use of an implementation artefact in the form of a prototype is used in order to evaluate the validity of the model is examined.

\section{Design Science Paradigm}

Henver \cite{Hevner2004} cites the two major paradigms that encapsulate the majority of Information System research as the behavioural and design sciences. Design science is in effect the science of artefacts and is therefore concerned with the design, development, improvement and understanding of artefacts for the purpose of solving problems and implementing IT systems \cite{March1995a}. Artefacts can be in the form of constructs, models, methods or instantiation \cite{Hevner2004}. These artefacts are created by the extension of existing kernel theories that are then applied, tested and modified through the experience, creativity and intuition of the researcher \cite{Walls1992}. Design science is also commonly used to solve identified organizational problems. In such cases, the resulting artefact allows researchers to understand the efficiency and feasibility of the artefact as solution to such a problem \cite{Hevner2004}.
 
March \cite{March1995a} dictates that design sciences does not pose theories but instead aim to create artefacts which are subject to two basic activities, build and evaluate. Therefore it is important to gather adequate information in order to construct an artefact according to specific requirements and then to evaluate this artefact using formal metrics. This paper aims at creating an artefact of the model type. This model should be a detailed architectural description of the interaction between different constructs that make up an effective multi-tenant SaaS application. Significant difficulties in design science are a result of the fact the evaluation such as performance of the artefact is related to the environment within which the artefact it operates \cite{March1995a}.


Combined with the subjectivity of the intended use and users of the actual artefact, makes evaluation a complex task. In order to evaluate the viability of this papers resulting (model) artefact, an (instantiation) artefact in the form of a prototype will be created. The prototype will also provide information on different other aspects of the implementation of the model such as feasibility, complexity and general application stability. Since the prototype artefact is not the target artefact examined by this paper but a means of evaluation, further specific tests to provide empirical data for analysis and evaluation of the prototype has not been included. However, several tests that could theoretically provide useful quantitative data includes stress testing/load testing the service to measure its ability to scale effectively. Additionally using monitoring tools such as Microsoft Application Insights \cite{AppINsight} and acceptance testing.

 
The development of a model artefact shares common traits with classical science in the sense that it attempts to simplify a complex phenomenon in order to be studies. Dodig-Crnkovic \cite{Dodig-crnkovic} suggests that the modeling process attempts to answer six questions relating to the phenomenon:

\begin{enumerate}
\item How should the model be constructed? Which features should be included and which neglected?
\item The appropriateness of the model should be measured in terms of its purpose, resolution and abstractness.
\item The exact features and behaviour of the model should be measured in comparison what is expected.
\item In which ways does the model itself differ from that of reality?
\item Is the model valid?
\item Which external or special constraints should be taken into consideration?
\end{enumerate}

These questions are used as guideline for constructing our model artefact and will be attempted to be answered in order to maintain some degree of relevance to the scientific method. Amaral \cite{Amaral2011} refers to design science as a "build" methodology which creates either a physical artefact or a software system in order to demonstrate that it is indeed possible to be created. Furthermore, Amaral reasons that in order for this methodology to be considered research, the construction of the artefact should not have been created before. Thus the artefact as such should contain new features or processes that have not been demonstrated before. Both the artefacts resulting from this paper attempts to apply features in a combination that has not been done in previous research. This is achieved using technologies that are extremely young (less than a month) old at time of writing and combining these with the specificity of the use case.


\section{Case Study}

In many research papers, a case study is used as the initial foundation by offering a complex research question while providing the proper context within which the problem should be solved \cite{Schell1992-cm}. In order to design an effective architecture or "model artefact", it is important to gather as many necessary design inputs in order to formalize the requirements and constraints that your architectural model should accommodate beforehand \cite{Microsoft_Patterns_Practices_Team2009-ek}. Case studies offer a comprehensive set of functional, non-functional, technological, and infrastructural requirements.
Case offers a much stronger context for implementation by others sharing similar a conditions and is therefore non-classical in the IT sciences. This thesis attempts to similarly create a complete context by providing domain information and system requirements used for creating the model artefact through a case study.


\section{Prototyping}

Prototyping is a powerful demonstration tool. A prototyped artefact aims to act as a pilot case or proof of concept showing that a concept or idea is viable in reality. Furthermore, a prototype aims to provide means for the collection of further specialized data in order to be reused for research purposes. In software engineering, prototyped software should be considered a special kind of software that does not aim to provide actual business value but instead provide grounds for exploration and data gathering \cite{Monperrus2013}. Due to this nature, prototyped software does not need to necessarily comply with common software design practices. Software written as prototype also only aims at working in preset limited amount of use cases and does not need to provide extended support or maintenance \cite{Monperrus2013}. Ideally software prototypes are tools for aligning ideas and attempting to simulate the ability to convert these ideas into reality. It is therefore important to note that the prototype created as part of this paper does not aim to provide specific implementation details or showcase specific design patterns or approaches. It also does not attempt to be a fully fledged system intended for future extension into an actual SaaS service but only exists as means of evaluating the architectural viability.


\section{Qualitative Approach}

A mostly qualitative effort has been used as means to evaluate the model artefacts design. This includes subjective knowledge provided by the implementation of the prototype in order to prove the models viability. This qualitative analysis of findings and experiences is used to further assess and evaluate the efficiency of design, implementation and suggest future improvements. As this paper is also a business oriented project compared to pure research based one, many of the results are based on the subjective view of its intended users and developers. Additionally, the model artefact is evaluated through using the SaaS maturity model as reference point and comparison. This maturity level is then combined with the requirements of the case study in order to determine if the design requirements of the model have been met. Finally the level of maturity reached by the architecture model is assessed and used in order to determine if the study has provided a relevant artefact as result.


\section{Conclusion}

In the next chapter we take a look into our XV Case study and explain the context used as baseline for the rest of the study.

